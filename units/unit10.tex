\providecommand{\main}{..}
\documentclass[\main/notes.tex]{subfiles}

\begin{document}
	\ifSubfilesClassLoaded{\setcounter{chapter}{9}}{}
	\chapter[Logic: Predicates and Proof Strategies]{Logic: Quantifiers, predicates and proof strategies}
		\section{Quantifiers and Predicates}
			\subsection[Universal Quantifier]{Universal Quantifier (FOR ALL)}
				\begin{definition}[width=0.8\textwidth]{Universal Quantifier}
					A \concept{universal quantifier} is written with the symbol $\forall$, meaning ``for all''.
				\end{definition}
				\begin{example}[width=0.6\textwidth]
					Examples would be:
					\begin{indentparagraph}
						``For all $x \in \mathbb{R}$ \ldots'' (written $\forall \: x \in \mathbb{R}$)\\
						``For every $x \in \mathbb{Z}$ \ldots'' (written $\forall \: x \in \mathbb{Z}$)
					\end{indentparagraph}
					The variable $x$ above is called a \concept{quantified variable}.
				\end{example}
				This can be considered a \emph{generalisation of conjunction (AND)}.
				\begin{example}[width=0.9\textwidth]
					Let $A$ = \{1, 2, 3\}. A declarative statement can then be made for the set:
					\begin{align*}
						\forall \: x \in A, x > 0
					\end{align*}
					This means the same thing as 
					\begin{align*}
						(1 > 0) \land (2 > 0) \land (3 > 0)
					\end{align*}
					This statement is true.
				\end{example}
				\pagebreak
			\subsection[Existential Quantifier]{Existential Quantifier (THERE EXISTS)}
				\begin{definition}[width=0.85\textwidth]{Existential Quantifier}
					An \concept{existential quantifier} is written with the symbol $\exists $, meaning ``there exists''.
				\end{definition}
				\begin{example}[width=0.65\textwidth]
					Examples would be:
					\begin{indentparagraph}
						``There exists an $x \in \mathbb{R}$ such that\ldots'' (written $\exists \: x \in \mathbb{R}$)\\
						``For some $x \in \mathbb{Z}$ \ldots'' (written $\exists \: x \in \mathbb{Z}$)
					\end{indentparagraph}
				\end{example}
				\begin{sidenote}{A quantified variable is a dummy variable}
					Any quantified variable can be replaced (everywhere it occurs) with another variable without changing the meaning.
					\begin{example}[hbox]
						$\forall \: x \in \mathbb{Z}^{+}, (x > 0) \equiv \forall \: y \in \mathbb{Z}^{+}, (y > 0)$
					\end{example}
				\end{sidenote}
				This can be considered a \emph{generalisation of disjunction (OR)}.
				\begin{example}[width=0.9\textwidth]
					Let $A$ = \{1, 2, 3\}. A declarative statement can then be made for the set:
					\begin{align*}
						\exists \: x \in A, x > 4
					\end{align*}
					This means the same thing as 
					\begin{align*}
						(1 > 4) \lor (2 > 4) \lor (3 > 4)
					\end{align*}
					This statement is false.
				\end{example}
				\begin{exercise}{Self Assessment Exercise \thechapter.3}
					\begin{questions}
						\item Write down the English equivalent of each of the following statements, and give an opinion on whether the statement is true.
							\begin{questions}[label=(\alph*)]
								\item $\exists y \in \mathbb{Q}, y = \sqrt{2}$\\
									{\answer There exists some rational number that is the square root of 2. This statement is false, as $\sqrt{2}$ is an irrational number.}
								\item $\forall x \in \mathbb{R}, 2x < x^{2}$\\
									{\answer For all real numbers $x$, $2x$ is less than $x^{2}$. This statement is false, as $2(0) \not < 0^{2}$}
								\item $\forall x \in \mathbb{Z}, x > 0$\\
									{\answer Every integer is greater than $0$. This statement is false, as $-1$ and $0$ are both integers.}
								\item $\exists x \in \mathbb{Z}^{+}, x = 0$\\
									{\answer There exists a positive integer that is equal to $0$. This statement is false, as $0$ is not a positive integer.}
							\end{questions}
					\end{questions}
				\end{exercise}
				\pagebreak
				\begin{exercise}{Self Assessment Exercise \thechapter.4}
					\begin{questions}
						\item Prove by means of truth tables that\\ $\lnot (p \land q) \equiv (\lnot p) \lor (\lnot q)$
							\begin{answer}
								\begin{center}
									\begin{tblr}{colspec={| c c | c c | c | c | c | c|}, row{1}={font=\bfseries}, row{odd[3]}={white}}
										\toprule
										$p$ & $q$ & $\lnot p$ & $\lnot q$ & $p \land q$ & $\lnot (p \land q)$ & $(\lnot p) \lor (\lnot q)$ & $\lnot (p \land q) \leftrightarrow (\lnot p) \lor (\lnot q)$\\
										\midrule
										T & T & F & F & T & F & F & T\\
										T & F & F & T & F & T & T & T\\
										F & T & T & F & F & T & T & T\\
										F & F & T & T & F & T & T & T\\
										\bottomrule
									\end{tblr}
								\end{center}
							\end{answer}
					\end{questions}
				\end{exercise}
				\begin{exercise}{Self Assessment Exercise \thechapter.5}
					\begin{questions}
						\item Prove by means of truth tables that\\ $\lnot (p \lor q) \equiv (\lnot p) \land (\lnot q)$
							\begin{answer}
								\begin{center}
									\begin{tblr}{colspec={| c c | c c | c | c | c | c|}, row{1}={font=\bfseries}, row{odd[3]}={white}}
										\toprule
										$p$ & $q$ & $\lnot p$ & $\lnot q$ & $p \lor q$ & $\lnot (p \lor q)$ & $(\lnot p) \land (\lnot q)$ & $\lnot (p \lor q) \leftrightarrow (\lnot p) \land (\lnot q)$\\
										\midrule
										T & T & F & F & T & F & F & T\\
										T & F & F & T & T & F & F & T\\
										F & T & T & F & T & F & F & T\\
										F & F & T & T & F & T & T & T\\
										\bottomrule
									\end{tblr}
								\end{center}
							\end{answer}
					\end{questions}
				\end{exercise}
			\subsection{Predicate}
				\begin{definition}{Predicate}
					A statement $P(x)$ is called a \concept{predicate} if it expresses some property of a variable $x \in A$, and returns either true or false depending on the value of $x$. $P(x)$ is true for any variable $x \in A$ that has the property, and $P(x)$ is false if $x$ does not have that property.
				\end{definition}
				\begin{sidenote}[width=0.65\textwidth]{A predicate is a boolean function}
					A predicate takes in a value, and either returns true or false.
				\end{sidenote}
			\subsection{Negation of Quantified Statements}
				If $P(x)$ is a predicate containing some variable $x$, then:
				\begin{enumerate}
					\item $\lnot \bigl(\forall \: x \in A, P(x)\bigr) \equiv \exists \: x \in A, \lnot P(x)$
					\item $\lnot \bigl(\exists \: x \in A, P(x)\bigr) \equiv \forall \: x \in A, \lnot P(x)$
				\end{enumerate}
			\begin{example}
				Determine the negation of the quantified statement ``$\forall x \in A, P(x) \lor Q(X)$''.
				\begin{align*}
					\lnot \bigl(\forall x \in A, P(x) \lor Q(X)\bigr) &\equiv \exists x \in A, \lnot \bigl(P(x) \lor Q(x)\bigr)\\
					& \equiv \exists \in A, \lnot P(x) \land \lnot Q(x)
				\end{align*}
			\end{example}
			\pagebreak
			\begin{exercise}{Self Assessment Exercise \thechapter.6}
				\begin{questions}
					\item Determine the negations of the following quantified statements: (Show all steps.)
						\begin{questions}
							\item $\forall x \in \mathbb{Z}^{+}, x > 3$
								\begin{answer}
									\begin{align*}
										\lnot (\forall x \in \mathbb{Z}^{+}, x > 3) &\equiv \exists x \in \mathbb{Z}^{+}, \lnot (x > 3)\\
										& \equiv \exists x \in \mathbb{Z}^{+}, x \leq 3
									\end{align*}
								\end{answer}
							\item $\exists x \in \mathbb{R}, 2x = x^{2}$
								\begin{answer}
									\begin{align*}
										\lnot (\exists x \in \mathbb{R}, 2x = x^{2}) & \equiv \forall x \in \mathbb{R}, \lnot(2x = x^{2})\\
										& \equiv \forall x \in \mathbb{R}, 2x \neq x^{2}
									\end{align*}
								\end{answer}
							\item $\forall x \in \mathbb{Z}, (x > 0) \lor (x^{2} > 0)$
								\begin{answer}
									\begin{align*}
										\lnot \bigl(\forall x \in \mathbb{Z}, (x > 0) \lor (x^{2} > 0)\bigr) & \equiv \exists x \in \mathbb{Z}, \lnot \bigl((x > 0) \lor (x^{2} > 0)\bigr)\\
										& \equiv \exists x \in \mathbb{Z}, \lnot (x > 0) \land \lnot (x^{2} > 0)\\
										& \equiv \exists x \in \mathbb{Z}, (x \leq 0) \land (x^{2} \leq 0)
									\end{align*}
								\end{answer}
							\item $\exists y \in \mathbb{Z}^{+}, (y \leq 10) \land (y \neq 0)$
								\begin{answer}
									\begin{align*}
										\lnot \bigl(\exists y \in \mathbb{Z}^{+}, (y \leq 10) \land (y \neq 0)\bigr) & \equiv \forall y \in \mathbb{Z}^{+}, \lnot \bigl((y \leq 10) \land (y \neq 0)\bigr)\\
										& \equiv \forall y \in \mathbb{Z}^{+}, \lnot (y \leq 10) \lor \lnot (y \neq 0)\\
										& \equiv \forall y \in \mathbb{Z}^{+}, (y > 10) \lor (y = 0)
									\end{align*}
								\end{answer}
							\item $\exists x \in A, P(x) \land Q(x)$
								\begin{answer}
									\begin{align*}
										\lnot \bigl(\exists x \in A, P(x) \lor Q(x)\bigr) & \equiv \forall x \in A, \lnot \bigl(P(x) \lor Q(x)\bigr)\\
										& \equiv \forall x \in A, \lnot P(x) \land \lnot Q(x)
									\end{align*}
								\end{answer}
							\item $\forall x \in \mathbb{Z}^{+}, (x \leq 3) \rightarrow (x^{3} \geq 1)$
								\begin{answer}
									\begin{align*}
										\lnot \bigl(\forall x \in \mathbb{Z}^{+} (x \leq 3) \rightarrow (x^{3} \geq 1)\bigr) &\equiv \lnot \bigl(\forall x \in \mathbb{Z}^{+}, \lnot (x \leq 3) \lor (x^{3} \geq 1)\bigr)\\
										& \equiv \exists x \in \mathbb{Z}^{+}, \lnot \bigl(\lnot (x \leq 3) \lor (x^{3} \geq 1)\bigr)\\
										& \equiv \exists x \in \mathbb{Z}^{+}, \lnot \lnot (x \leq 3) \land \lnot (x^{3} \geq 1)\\
										& \equiv \exists x \in \mathbb{Z}^{+}, (x \leq 3) \land \lnot (x^{3} \geq 1)\\
										& \equiv \exists x \in \mathbb{Z}^{+}, (x \leq 3) \land (x^{3} < 1)
									\end{align*}
								\end{answer}
						\end{questions}
				\end{questions}
			\end{exercise}
			\pagebreak
			\begin{exercise}{Self Assessment Exercise \thechapter.7}
				
				\begin{questions}
					\item For each of (a) to (d) in the previous exercise, determine whether the original statement is true, whether the negation is true, or if neither of the two is true.
					\begin{questions}[label=(\alph*)]
						\item {\answer The original statement is false, as $1$, $2$ and $3$ are positive integers. The negation is true.}
						\item {\answer The original statement is true, as when $x = 2$, $2(2) = (2)^{2}$. The negation is false, as there is an element.}
						\item {\answer The original statement is false, as $0 \not > 0$ and $0^{2} \not > 0$. The negation is true, if $x = 0$.}
						\item {\answer The original statement is true for any positive integer less than 10. The negation is false, as not all positive integers are greater than 10.}
					\end{questions}
				\end{questions}
			\end{exercise}
		\section{Proof Strategies}
				Given some statement ``if $p$, then $q$'', there are different ways to prove it.
			\subsection{Direct Proof}
				Assume that $p$ is true, and then reason step-by-step to show that $q$ is true.
				\begin{example}
					Prove that the following statement is true for all $x \in \mathbb{R}$:
					\begin{align*}
						\text{If } x^{2} - 4x + 3 < 0 \text{, then } x > 0
					\end{align*}
					Start by assuming that $x^{2} - 4x + 3 < 0$ is true.
					\begin{proof}
						Assume $x^{2} - 4x + 3 < 0$. That is:
						\begin{align*}
							x^{2} - 4x + 3 &< 0\\
							(x - 3)(x - 1) &< 0 \tag*{(by factorisation)}
						\end{align*}
						That means either 
							\begin{enumerate}[label=(\roman*)]
								\item $(x - 3) > 0$ and $(x - 1) < 0$ (plus times minus gives minus), or
								\item $(x - 3) < 0$ and $(x - 1) > 0$ (minus times plus gives minus)
							\end{enumerate}
						For (i):
						$ \begin{alignedat}[t]{3}
								& & (x - 3) &> 0 \quad \text{ and }\quad  (x - 1) \: &< 0\\
								& \Rightarrow \quad & x &> 3 \quad \text{ and } \quad x &< 1
							\end{alignedat} $
						\begin{indentparagraph}
							There is no $x$ that this can be true for.
						\end{indentparagraph}
						For (ii):
						$ \begin{alignedat}[t]{3}
								& & (x - 3) &< 0 \quad \text{ and }\quad  (x - 1) \: &> 0\\
								& \Rightarrow \quad & x &< 3 \quad \text{ and } \quad x &> 1
						\end{alignedat} $
							\begin{indentparagraph}
								This shows $1 < x < 3$, or $x \in (1, 3)$.\\
								Therefore, $x < 0$
							\end{indentparagraph}
					\end{proof}
				\end{example}
			\pagebreak
			\subsection[Proof By Contradiction]{Proof By Contradiction (\emph{Reductio Ad Absurdum})}
				Assume that $p$ is true. Then assume that $q$ is false, and use step-by-step reasoning until there is a contradiction. If there is a contradiction, that means that $q$ must be true.
				\begin{example}
					Prove that the following statement is true for all $x \in \mathbb{R}$:
					\begin{align*}
						\text{If } x^{2} - 4x + 3 < 0 \text{, then } x > 0
					\end{align*}
					Start by assuming that $x^{2} - 4x + 3 < 0$ is true.
					\begin{proof}
						Assume $x^{2} - 4x + 3 < 0$.\\
						If the antecedent is true, then either the consequent is true or the consequent is false.\\
						Assume that the consequent is false, i.e. assume that $x \not > 0$, that is $x \leq 0$.
						\begin{tabbing}
							If $\qquad$ \=$x \leq 0$,\\
							Then \>$-4x \geq 0$ (minus times minus gives plus)\\
							And \>$x^{2} + 3 > 0$\\
							So \>$x^{2} - 4x + 3 > 0$
						\end{tabbing}
						However, this contradicts the original assumption. Therefore, $x \leq 0$ cannot be true.\\
						Therefore, $x > 0$.
					\end{proof}
				\end{example}
			\subsection{Proof By Contrapositive}
				\begin{definition}{Contrapositive}
					The \concept{contrapositive} of $p \rightarrow q$ is $\lnot q \rightarrow \lnot p$. These two statements are logically equivalent to each other.
				\end{definition}
				\begin{example}
					Prove that the following statement is true for all $x \in \mathbb{R}$:
					\begin{align*}
						\text{If } x^{2} - 4x + 3 < 0 \text{, then } x > 0
					\end{align*}
					To use the contrapositive, swap the two statements around, and negate them:
					\begin{proof}
						To use the contrapositive, prove:
						\begin{align*}
							\text{If } \lnot(x > 0) \text{, then } \lnot(x^{2} - 4x + 3 < 0).
						\end{align*}
						Assume $\lnot (x > 0)$ is true, i.e. $x \leq 0$.\\
						Factorise the consequent:
						\begin{align*}
							x^{2} - 4x + 3 = (x - 3)(x - 1)
						\end{align*}
						As $x \leq 0$, $(x - 3) \leq 0$ and $(x - 1) \leq 0$.
						\begin{tabbing}
							If $\qquad$ \= $(x - 3) \leq 0$ and $(x - 1) \leq 0$,\\
							Then \> $(x - 3)(x - 1) \geq 0$ (minus times minus gives plus)\\
							i.e. \> $x^{2} - 4x + 3 \geq 0$\\
							i.e. \> $\lnot (x^{2} - 4x + 3 < 0)$
						\end{tabbing}
					\end{proof}
				\end{example}
				\begin{sidenote}[width=0.8\textwidth]{The contrapositive is not the same as the converse}
					The \concept{converse} of a statement just swaps them around. This is not the same as the contrapositive.
					\begin{example}[width=0.6\textwidth]
						Given a statement, $p \rightarrow q$.
						\begin{description}
							\item[Converse] $q \rightarrow p$
							\item[Contrapositive] $\lnot q \rightarrow \lnot p$ 
						\end{description} 
					\end{example}
				\end{sidenote}
			\subsection{Proofs Involving Quantifiers}
				In order to apply a proof to a quantified statement over an infinite set $A$, for example $\forall \: x \in A, P(x)$, think of the statement as $x \in A \rightarrow P(x)$.
				\begin{example}[width=0.38\textwidth]
					Prove the statement
					\begin{align*}
						\forall \: x \in \mathbb{R}, x^{2} + 1 > 0.
					\end{align*}
					\begin{proof}
						$ $
						\begin{tabbing}
							If $\qquad$ \=$x \in \mathbb{R}$,\\
							Then \>$x^{2} \geq 0$,\\
							So \> $x^{2} + 1 \geq 1$\\
							i.e. \> $x^{2} + 1 > 0$
						\end{tabbing}
					\end{proof}
				\end{example}
				To disprove a statement, prove that its \emph{negation} is true. If this is a statement such as $\forall \: x \in A, P(x)$, the negation is $\exists \: x \in A, \lnot P(x)$. This shows that in order to disprove the statement, one needs to simply find a counterexample.
				\begin{example}[width=0.8\textwidth]
					Show using a \emph{counterexample} that this statement is not true:
					\begin{align*}
						\forall \: x \in \mathbb{R}, x^{2} - 4x > 0
					\end{align*}
					\begin{proof}
						To disprove $\forall \: x \in \mathbb{R}, x^{2} - 4x > 0$, one needs to prove that:
						\begin{align*}
							\exists \: x \in \mathbb{R}, x^{2} - 4x \geq 0
						\end{align*}
						One could choose $x = 0$.\\
						Then \begin{align*}
							x^{2} - 4x &= (0)^{2} - 4(0)\\
							&= 0\\
							& \not > 0
						\end{align*}
					\end{proof}
				\end{example}
			\pagebreak
			\subsection{Vacuous Proofs}
				The truth table for an implication shows that if the antecedent is false, then the statement is always true.\\
				Using the above, if you can show that the conditional statement is false, then the statement is \concept{vacuously true}.
				\begin{example}[width=0.7\textwidth]
					Prove that:
					\begin{align*}
						\emptyset \subseteq X
					\end{align*}
					To prove the above statement, we need to show that:
						\begin{align*}
							\text{If } x \in \emptyset \text{, then } x \in X
						\end{align*}
					\begin{proof}
						$ $\\
						$\emptyset$ is an empty set,\\
						so ``$x \in \emptyset$'' is false,\\
						therefore ``if $x \in \emptyset$, then $x \in X$'' is \concept{vacuously true}.
					\end{proof}
				\end{example}
				\begin{example}
					Let $S$ be a relation on $\{a, b, c, d\}$, where $S = \bigl\{(a, b), (a, d)\bigr\}$. Prove that $S$ is transitive.
					\begin{proof} For a set $S$ to be transitive, whenever $(x, y) \in S$ and $(y, z) \in S$, then $(x, z) \in S$.\\
						There are no two pairs of the form $(x, y)$ and $(y, z)$ in $S$,\\
						so it is \concept{vacuously true} that $S$ is transitive.
					\end{proof}
				\end{example}
				\begin{exercise}{Self Assessment Exercise \thechapter.10}
					\begin{questions}
						\item Prove each of the following statements by direct proof, contrapositive and contradiction respectively.
							\begin{questions}
								\item If $x^{2} - 3x + 2 < 0$, then $x > 0$.
									\begin{answer}
										\begin{proof}[Direct Proof]
											Assume that $x^{2} - 3x + 2 < 0$ is true. That is $(x - 1)(x - 2) < 0$, by factorisation.\\
											So either
											\begin{enumerate}[label=(\roman*)]
												\item 
													$ \begin{alignedat}[t]{3}
														& & (x - 1) &< 0 \quad \text{ and } \quad (x - 2) \: &> 0\\
														& \Rightarrow & x &< 1 \quad \text{ and } \quad x &> 2 
													\end{alignedat} $
													\begin{indentparagraph}
														There is no $x$ that this can be true for.
													\end{indentparagraph}
												\item 
													$ \begin{alignedat}[t]{3}
														& & (x - 1) &> 0 \quad \text{ and } \quad (x - 2) \: &< 0\\
														& \Rightarrow & x &> 1 \quad \text{ and } \quad x &< 2 
													\end{alignedat} $
													\begin{indentparagraph}
														So $1 < x < 2$.
													\end{indentparagraph}
											\end{enumerate}
											As $1 < x < 2$, $x > 0$.
										\end{proof}
										\begin{proof}[Contrapositive]
											To show: If $x \leq 0$, then $x^{2} - 3x + 2 \geq 0$.\\
											Suppose $x \leq 0$. Then $-3x \geq 0$. And $x^{2} + 2 > 0$.\\
											So $x^{2} - 3x + 2 \geq 0$
										\end{proof}
										\begin{proof}[Contradiction]
											Assume that $x^{2} - 3x + 2 < 0$. Suppose that $x \leq 0$.\\
											If $x \leq 0$, then $-3x \geq 0$. And $x^{2} + 2 \geq 0$.\\
											So $x^{2} - 3x + 2 \geq 0$.\\
											But that contradicts the original assumption.\\
											So it must be the case that $x > 0$
										\end{proof}
									\item If $x^{2} - x - 6 > 0$, then $x \neq 1$.
										\begin{proof}[Direct Proof]
											Assume that $x^{2} - x - 6 > 0$. That is, $(x - 3)(x + 2) > 0$.\\
											So either
											\begin{enumerate}[label=(\roman*)]
												\item
													$ \begin{alignedat}[t]{3}
														& & (x - 3) &> 0 \quad \text{ and } \quad (x + 2) \: &> 0\\
														& \Rightarrow & x &> 3 \quad \text{ and } \quad x &> -2 
													\end{alignedat} $
													\begin{indentparagraph}
														So $x > 3$.
													\end{indentparagraph}
												\item
													$ \begin{alignedat}[t]{3}
														& & (x - 3) &< 0 \quad \text{ and } \quad (x + 2) \: &< 0\\
														& \Rightarrow & x &< 3 \quad \text{ and } \quad x &< -2 
													\end{alignedat} $
													\begin{indentparagraph}
														So $x < -2$.
													\end{indentparagraph}
												\end{enumerate}
												So either $x > 3$ or $x < -2$.\\
												So $x \neq 1$.
										\end{proof}
										\begin{proof}[Contrapositive]
											To show: If $x = 1$, then $x^{2} - x - 6 \leq 0$.\\
											Suppose $x = 1$. Then $ \begin{aligned}[t]
												x^{2} - x - 6 &= (1)^{2} - 1 - 6\\
												&= 1 - 1 - 6\\
												&= -6
											\end{aligned}$ \\
											As $-6 < 0$, $x^{2} - x - 6 \leq 0$.
										\end{proof}
										\begin{proof}[Contradiction]
											Assume that $x^{2} - x - 6 > 0$.\\
											Suppose that $x = 1$. Then $ \begin{aligned}[t]
												x^{2} - x - 6 &= (1)^{2} - 1 - 6\\
												&= 1 - 1 - 6\\
												&= -6
											\end{aligned} $\\
											So $x^{2} - x - 6 < 0$.\\
											But that contradicts the original assumption.\\
											So it must be the case that $x \neq 1$.
										\end{proof}
									\end{answer}
								\pagebreak
								\item For all $a, b \in \mathbb{Z}$, if $a + b$ is odd, then exactly one of $a$ or $b$ is odd.
									\begin{answer}
										\begin{proof}[Direct Proof]
											Assume that $a + b$ is odd (where $a, b \in \mathbb{Z}$).\\
											Then $a + b = 2n + 1$ for some integer $n$.
											Then either
											\begin{enumerate}[label=(\roman*)]
												\item $a$ is even.\\
													Suppose $a$ is even. Then $a = 2k$ for some integer $k$. 
													\begin{alignat*}{2}
														& & a + b &= 2k + b\\
														& & 2k + b &= 2n + 1\\
														& \Rightarrow \quad &b &= 2n + 1 - 2k\\
														& \Rightarrow \quad &b &= 2(n - k) + 1
													\end{alignat*}
													So $b$ is odd.
												\item $a$ is odd.\\
													Suppose $a$ is odd. Then $a = 2k + 1$ for some integer $k$. 
													\begin{alignat*}{2}
														& & a + b &= 2k + 1 + b\\
														& & 2k + 1 + b &= 2n + 1\\
														& \Rightarrow \quad &b &= 2n + 1 - 2k - 1\\
														& \Rightarrow \quad &b &= 2(n - k)
													\end{alignat*}
													So $b$ is even.
											\end{enumerate}
											So if $a$ is even, then $b$ is odd. If $a$ is odd, then $b$ is even.\\
											So exactly one of $a$ or $b$ is odd.
										\end{proof}
										\begin{proof}[Contrapositive]
											To show: If both $a$ and $b$ are odd, or both $a$ and $b$ are even, then $a + b$ is not odd.\\
											There are two cases:
											\begin{enumerate}[label=(\roman*)]
												\item Suppose $a$ and $b$ are both odd. Then $a = 2n + 1$ for some integer $n$, and $b = 2k + 1$ for some integer $k$.\\
													So $ \begin{aligned}[t]
														a + b &= (2n + 1) + (2k + 1)\\
														&= 2n + 2k + 2\\
														&= 2(n + k + 1)
													\end{aligned}$\\
													So $a + b$ is even, i.e. $a + b$ is not odd.
												\item Suppose $a$ and $b$ are both even. Then $a = 2n$ for some integer $n$ and $b = 2k$ for some integer $k$.\\
													So $ \begin{aligned}[t]
														a + b &= 2n + 2k\\
														&= 2(n + k)
													\end{aligned}$\\
													So $a + b$ is even, i.e. $a + b$ is not odd.
											\end{enumerate}
											So, if it is not the case that exactly one of $a$ or $b$ is odd, then $a + b$ is not odd.
										\end{proof}
										\pagebreak
										\begin{proof}[Contradiction]
											Assume that $a + b$ is odd (where $a, b \in \mathbb{Z}$).\\
											Then either exactly one of $a$ and $b$ is odd, or that is not the case.\\
											If it is not the case, then either $a$ and $b$ are both odd, or $a$ and $b$ are both even.
											\begin{enumerate}[label=(\roman*)]
												\item Suppose $a$ and $b$ are both odd. Then $a = 2n + 1$ for some integer $n$, and $b = 2k + 1$ for some integer $k$.\\
													So $ \begin{aligned}[t]
														a + b &= (2n + 1) + (2k + 1)\\
														&= 2n + 2k + 2\\
														&= 2(n + k + 1)
													\end{aligned}$\\
													So $a + b$ is even, which contradicts the original assumption.
												\item Suppose $a$ and $b$ are both even. Then $a = 2n$ for some integer $n$ and $b = 2k$ for some integer $k$.\\
													So $ \begin{aligned}[t]
														a + b &= 2n + 2k\\
														&= 2(n + k)
													\end{aligned}$\\
													So $a + b$ is even, which contradicts the original assumption.
											\end{enumerate}
											Therefore, it must be the case that exactly one of $a$ and $b$ is odd.
										\end{proof}
									\end{answer}
								\item For all $x \in \mathbb{Z}$, if $x$ is even, then $x^{2} + 4x + 2$ is even.
									\begin{answer}
										\begin{proof}[Direct Proof]
											Assume that $x$ is even, where $x \in \mathbb{Z}$. If $x$ is even, then $x = 2k$ for some integer $k$.\\
											So $\begin{aligned}[t]
												x^{2} + 4x + 2 &= (2k)^{2} + 4(2k) + 2\\
												&= 4k^{2} + 8k + 2\\
												&= 2(2k^{2} + 4k + 1)
											\end{aligned} $\\
											So $x^{2} + 4x + 2$ is even.
										\end{proof}
										\begin{proof}[Contrapositive]
											To show: If $x^{2} + 4x + 2$ is odd, then $x$ is odd.\\
											Suppose $x^{2} + 4x + 2$ is odd. Then $x^{2} + 4x + 2 = 2k + 1$ for some integer $k$\\
											So $\begin{alignedat}[t]{2}
												& & x^{2} + 4x + 2 &= 2k + 1\\
												& \Rightarrow \quad & x^{2} + 4x + 4 &= 2k + 1 + 2 \qquad \text{(Complete the square)}\\
												& \Rightarrow \quad & (x + 2)(x + 2) &= 2k + 2 + 1\\
												& \Rightarrow \quad & (x + 2)(x + 2) &= 2(k + 1) + 1
											\end{alignedat}$\\
											So, as $(x + 2)(x + 2)$ is odd, that means that $x + 2$ must be odd, as the product of the integers is odd iff both integers are odd.\\
											So $x + 2 = 2n + 1$ for some integer $n$.\\
											So $x = 2n + 1 - 2$, so $x = 2n - 1$.\\
											So $x$ is odd.\\
											So, if $x^{2} + 4x + 2$ is odd, then $x$ is odd.\\
											So, if $x$ is even, then $x^{2} + 4x + 2$ is even.
										\end{proof}
										\pagebreak
										\begin{proof}[Contradiction]
											Assume $x$ is even. Then either $x^{2} + 4x + 2$ is even, or odd.\\
											Suppose that $x^{2} + 4x + 2$ is odd. Then $x^{2} + 4x + 2 = 2k + 1$ for some integer $k$\\
											So $\begin{alignedat}[t]{2}
												& & x^{2} + 4x + 2 &= 2k + 1\\
												& \Rightarrow \quad & x^{2} + 4x + 4 &= 2k + 1 + 2 \qquad \text{(Complete the square)}\\
												& \Rightarrow \quad & (x + 2)(x + 2) &= 2k + 2 + 1\\
												& \Rightarrow \quad & (x + 2)(x + 2) &= 2(k + 1) + 1
											\end{alignedat}$\\
											So, as $(x + 2)(x + 2)$ is odd, that means that $x + 2$ must be odd, as the product of the integers is odd iff both integers are odd.\\
											So $x + 2 = 2n + 1$ for some integer $n$.\\
											So $x = 2n + 1 - 2$, so $x = 2n - 1$.\\
											So $x$ is odd.\\
											But this contradicts the original assumption, so it must be the case that $x^{2} + 4x + 2$ is even.
										\end{proof}
									\end{answer}
								\item If $n$ is a multiple of $3$, then $n^{3} + n^{2}$ is a multiple of $3$
									\begin{answer}
										\begin{proof}[Direct Proof]
											Assume that $n$ is a multiple of $3$. Then $n = 3k$ for some integer $k$.\\
											So $\begin{aligned}[t]
												n^{3} + n^{2} &= (3k)^{3} + (3k)^{2}\\
												&= 27k^{3} + 9k^{2}\\
												&= 3(9k^{3} + 3k^{2})
											\end{aligned} $\\
											Therefore $n^{3} + n^{2}$ is a multiple of $3$.
										\end{proof}
										\begin{proof}[Contrapositive]
											To show: If $n^{3} + n^{2}$ is not a multiple of $3$, then $n$ is not a multiple of $3$.\\
											Suppose that $n^{3} + n^{2}$ is not a multiple of $3$. Then $n^{3} + n^{2} = 3k + 1$ for some integer $k$.\\
											If $n^{3} + n^{2} = 3k + 1$, then $n(n^{2} + n) = 3k + 1$.\\
											So neither $n$, nor $n^{2} + n$ are multiples of $3$.\\
											So $n$ is not a multiple of $3$.\\
											Therefore, if $n$ is a multiple of $3$, then $n^{3} + n^{2}$ is a multiple of $3$.
										\end{proof}
										\begin{proof}
											Suppose $n$ is a multiple of $3$. Then either $n^{3} + n^{2}$ is a multiple of $3$, or not.\\
											If $n^{3} + n^{2}$ is not a multiple of $3$, then $n^{3} + n^{2} = 3k + 1$ for some integer $k$.\\
											If $n^{3} + n^{2} = 3k + 1$, then $n(n^{2} + n) = 3k + 1$.\\
											So neither $n$, nor $n^{2} + n$ are multiples of $3$.\\
											So $n$ is not a multiple of $3$.\\
											But this contradicts the original assumption.\\
											So, if $n$ is a multiple of $3$, then $n^{3} + n^{2}$ is a multiple of $3$.
										\end{proof}
									\end{answer}
							\end{questions}
						\item Provide a counterexample to show that the statement \\ ``If $x > 0$, then $x^{2} - 3x + 1 < 0$'' is not true for all integers $x > 0$.\\
							\begin{answer}
								Let $x = 4$. Then $ \begin{aligned}[t]
									x^{2} - 3x + 1 &= (4)^{2} - 3(4) + 1\\
									&= 16 - 12 + 1\\
									&= 5 \not < 0
								\end{aligned}$
							\end{answer}
					\end{questions}
				\end{exercise}
				\vspace*{\parskip}
				\ifSubfilesClassLoaded{%
					\vbox{\rulechapterend}}{\vspace*{\parskip}\rulebookend}
\end{document}
