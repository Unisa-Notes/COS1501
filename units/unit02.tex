\documentclass[../notes.tex]{subfiles}

\begin{document}
	\ifSubfilesClassLoaded{\setcounter{chapter}{1}}{}
	\chapter{Rational and Real Numbers}
		\section{Rational Numbers}
			Denoted $\mathbb{Q}$, the set of all numbers in the form $\frac{p}{q}$ where $p$ and $q$ are integers are $q$ is not zero.
			\subsection{Multiplicative Inverses}
				For every non-zero rational number $x$ there exists a rational number called the \textbf{multiplicative inverse}, denoted $\frac{1}{x}$ such that $\bigl(x\bigr)\left(\frac{1}{x}\right) = 1$.\\
				This can also be written:\\
				\-\hspace{2em}For every non-zero rational number $x$ there exists a rational number $y$ such that $xy = 1$.
		\section{Real Numbers}
			Denoted $\mathbb{R}$, the combination of the rational and irrational numbers.
		\section{Number Systems Heirarchy}
			\begin{align*}
				\mathbb{C} > \mathbb{R} > \mathbb{Q}' > \mathbb{Q} > \mathbb{Z} > \mathbb{Z}^{\geq} > \mathbb{Z}^{+}
			\end{align*}
			\pagebreak
			\begin{exercise}{Self-Assessment Exercise (Activity \thechapter.8)}
				\begin{enumerate}
					\item \textbf{Define the words "even" and "odd" for positive integers}\\
						An integer is \textbf{even} if it is a multiple of $2$. An integer is \textbf{odd} if it is not even.
					\item \textbf{Is it the case that $m + (n\cdot k) = (m + n)(m + k)$ for all positive integers $m$, $n$ and $k$?}\\
						No. In order to show this, use a \textit{counterexample}. \rule{0pt}{11pt} \vspace*{-18pt}
						\begin{proof}
							Let $m = 1$, $n = 2$, $k = 3$. Then
							\begin{align*}
								m + (n \cdot k) &= 1 + \bigl((2)(3)\bigr)\\
								&= 1 + 6\\
								&= 7\\
								(m + n)(m + k) &= (1 + 2)(1 + 3)\\
								&= (3)(4)\\
								&= 12\\
								7 &\neq 12\\
								\therefore m + (n \cdot k) &\neq (m + n)(m + k) \qedhere
							\end{align*}
						\end{proof}
					\item \textbf{Are there any even prime numbers besides $2$?}\\
					No. \rule{0pt}{11pt} \vspace*{-18pt}
					\begin{proof}
						Let $m$ be an even number that is not $2$.\\
						Then $m = 2k$ where $k$ is some real number.\\
						Therefore $2$ and $k$ are factors of $m$.\\
						Therefore $m$ is not a prime number.
					\end{proof}
					\item \textbf{If $m$ and $n$ are even, is $m + n$ even?}\\
					Yes. \rule{0pt}{11pt} \vspace*{-18pt}
					\begin{proof}
						Let $m$ and $n$ be even numbers.\\
						Then $m = 2j$, $n = 2k$, where $j$ and $k$ are some real numbers.\\
						Then
						\begin{align*}
							m + n &= 2j + 2k\\
							&= 2(j + k)
						\end{align*}
						As the sum of the two numbers is a multiple of $2$, $m + n$ is even.
					\end{proof}
					\item \textbf{If $m$ and $n$ are odd, is $m \cdot n$ odd?}\\
					Yes. \rule{0pt}{11pt} \vspace*{-18pt}
					\begin{proof}
						Let $m$ and $n$ be odd numbers.\\
						Then $m = 2j + 1$, $n = 2k + 1$, where $j$ and $k$ are some real numbers.
						Then
						\begin{align*}
							m\cdot n &= (2j + 1)(2k + 1)\\
							&= 4jk + 2k + 2j + 1\\
							&= 2(2jk + k + j) + 1
						\end{align*}
						$\therefore m \cdot n$ is odd. 
					\end{proof}
				\end{enumerate}
			\end{exercise}
			\noindent\rule{\textwidth}{0.4pt}
\end{document}